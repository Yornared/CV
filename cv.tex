\documentclass[10pt,a4paper]{article}

% Packages
\usepackage[margin=2cm]{geometry} % Simple margin setup
\usepackage{hyperref} % For links
\usepackage{enumitem} % For lists
\usepackage{fontawesome5} % For icons
\usepackage{parskip} % Removes indentation and adds spacing between paragraphs
\usepackage{titlesec} % For section formatting
\usepackage{charter} % Font

% Section formatting
\titleformat{\section}{\bfseries\large}{}{0pt}{}[\vspace{1pt}\rule{\linewidth}{0.4pt}]
\titlespacing{\section}{0pt}{0.3cm}{0.2cm}

% List formatting
\newlist{highlights}{itemize}{1}
\setlist[highlights]{label=\textbullet, left=0pt, itemsep=0.05cm, topsep=0.05cm}

% Header
\newcommand{\header}[4]{
    \begin{center}
        {\LARGE \textbf{#1}}\\[2pt]
        #2 \textbullet\ 
        \href{mailto:#3}{#3} \textbullet\ 
        \href{tel:#4}{#4} \\
    \end{center}
    \vspace{0.2cm}
}

\begin{document}

% ===== Header =====
\header{Kiran De Roy}{Betekom}{yornared@gmail.com}{+32 487 58 92 98}{\href{https://linkedin.com/in/kiran-de-roy-57a440269}{\faLinkedin\ LinkedIn}}

% ===== Education =====
\section{Education}
\textbf{KAMSA Aarschot} \hfill Sept 2014 -- June 2020\\
Science and Mathematics


\textbf{KU Leuven: Group T} \hfill Sept 2020 -- Feb 2025\\
BSc in Electronics-ICT: Smart Electronics and Software  
\begin{highlights}
    \item Coursework: Operating Systems, Object-Oriented Programming, Web Development
\end{highlights}


\textbf{KU Leuven: Group T} \hfill Sept 2024 -- June 2025\\
MSc in Electronics-ICT: Software Systems  
\begin{highlights}
    \item Coursework: C++, Machine Learning, Cybersecurity, Distributed Systems
\end{highlights}

% ===== Experience =====
\section{Experience}
\textbf{Student Job, Delhaize} \hfill July 2019 -- July 2021  
\begin{highlights}
    \item Operated the cash register
    \item Restocked shelves
\end{highlights}

% ===== Research =====
\section{Research}
\textbf{Master's Thesis – Real-Time Pelvic Rotation Detection Using Machine Learning} \hfill Sept 2024 -- June 2025  
Developed machine learning models for real-time pelvic rotation detection to address lower back pain, achieving 95\% weighted F1-score using MS-TCN++ temporal model on IMU sensor data.  

\textbf{Technologies:} Python, Scikit-learn, Pandas, NumPy, Tensorflow, PyTorch, CUDA  

Supervisor: \textit{Bart Vanrumste} \\
Daily Supervisor: \textit{Meixing Liao} \\
Collaborator: Stijn Avaux

% ===== Projects =====
\section{Projects}
\textbf{Automatic Chessboard with Lichess Integration} \hfill \href{https://github.com/nicogutz/ESP32_BoardCode/tree/master}{\faGithub\ Source Code}  
\begin{highlights}
    \item Built an automatic chessboard powered by ESP32-S3 using CoreXY mechanism and electromagnets
    \item Integrated with Lichess API via React web app
    \item Tools: ESP-IDF (C), React, Stepper Motor Drivers, CoreXY Kinematics
\end{highlights}

\textbf{Student Freelancing SaaS Application: UniGigs} \hfill \href{https://github.com/SaaS-Team-1/FreelanceApp}{\faGithub\ Source Code}  
\begin{highlights}
    \item SaaS web app for students to offer and find freelance gigs
    \item Authentication, file storage via Firebase, payments via Stripe
    \item Tools: React, TypeScript, Firebase
\end{highlights}

\textbf{Distributed Food Ordering System} \hfill \href{https://github.com/Yornared/Food-ordering-webshop}{\faGithub\ Source Code}  
\begin{highlights}
    \item Microservices-based food ordering platform with 2PC transactions
    \item Designed REST APIs, monitoring services, deployed to Azure
    \item Tools: Java, Spring Boot, REST APIs, Maven, Azure, GitHub Actions
\end{highlights}

% ===== Technologies =====
\section{Technologies}
\textbf{Languages:} Java, C++, C, PHP, SQL, JavaScript, Python\\
\textbf{Technologies:} Symfony, Git, React, Firebase, QTCreator, Tensorflow

\end{document}
